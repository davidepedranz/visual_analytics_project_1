\section{Data Description}
\label{sec:data}

The datasets used by Gapminder are public and available for free online \footnote{\url{http://www.gapminder.org/data/}}.

The datasets represent an indicator measured over the entire world and different time periods.
Each dataset has \num{3} dimensions: time, geographical location and indicator.

Time is measured in years.
The ``less than'' $<$ relationship in the set of natural numbers $\mathbb{N}$ defines a total order between years.
It is also possible to sum or subtract a natural number for a given year and the result is still a year.
Thus, time is a discrete attribute.

The geographical location is expressed in countries.
It does not exists a natural ordering of countries, thus the attribute type is categorical.
Gapminder has data about \num{244} different countries in the world.

The last dimension is different for each dataset.
The attribute type of each indicator is different and depends on the indicator itself.
An example of categorical attributes is ``Democracy score''.
Some categorical attributes represent meta-information about other indicators in Gapminder and are suggested for advanced usages only.

% TODO: discrete

Most indicator are quantitative, such as ``Income per person (GDP/capita, PPP\$ inflation-adjusted)'' or ``Life expectancy (years)''.

Each dataset in Gapminder can be represented as a table:
the x-axis represent the time in years, the y-axis represent the country and the data in each cell represent the indicator.

An alternative representation of the data is a set of tuples
each tuple has two dimensions that represent time and country respectively, and an extra dimension for each indicator.

Not all datasets in Gapminder are complete:
most indicators are defined for a small time period (few years) and have missing data for some country and / or some particular year.
