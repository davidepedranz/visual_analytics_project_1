\section{Problem Description}
\label{sec:problem}

\subsection{Questions}
Gapminder tries to answer to many questions.
We start from low-level concrete questions, then try to use them to understand the real problem that the tool aims to solve. 

\pagebreak

\paragraph{Concrete Questions}
\begin{itemize}

    % population size
    \item Where are the countries with the largest population located?
    \item Which continent has the largest population? And the smallest?

    % life expectation
    \item What is the life expectation of the world population?
    \item Which countries have the highest life expectation? Which have the lowest?
    \item How did the life expectation of people living in Europe change from 1800 to today?
    \item Was there any drop in the average life expectation during the world wars? In which countries?

    % income
    \item Which countries have the highest income? Which have the lowest income?
    \item What is the average income of the population in Africa?
    \item Did the average income of Asia increased from 1800 to today?  

    % correlation
    \item Is there any country with high income but low life expectation? What about the opposite?
    \item Are the average incomes of small countries above or under the world average?
    \item Where are the countries with high income and life expectation located?
    \item Do small countries have a longer life expectation?

\end{itemize}

\paragraph{Higher-level Questions}
\begin{itemize}

    % population
    \item What is the population distribution in the world?
    \item Is the size of the population changing over time? Where is it changing? Where is it not?

    % life expectation
    \item What is the global trend of the life expectation? Is it different for different continents?
    \item Do countries closed to each other have similar life expectations?    

    % income
    \item What is the global trend of the average income?
    \item Do countries closed to each other have similar incomes?    

    % life vs income
    \item Is there a correlation between life expectation and income today?    
    \item Did the correlation between life expectation and income change over time?

    % size vs ...
    \item Does the size of a population influence the life expectation?
    \item Is there a correlation between population size and average income?

    % outliers
    \item Are there outliers countries in the different continents (in terms of life expectation and income)? How are they different from the other countries?

\end{itemize}

\paragraph{Summary}
\begin{itemize}
    \item What is the world situation in terms of population, average income, life expectation?
    \item How did the world situation change over the last 2 centuries?
    \item Can you predict the trend for the next years?
\end{itemize}

\vspace{3mm}

The central element in Gapminder is time, which is present in all visualization.
The tool visualizes changes and trends for the selected indicators for different countries over time.
Even though it is possible to build visualizations without where time is not used, Gapminder is not designed for that.
We will discuss in greater detail of time is visualized in Gapminder in \cref{sec:visual_design}.


\subsection{Users}
Each indicator in Gapminder has time and geographical information.
Indicators range among very different categories and the tool makes trivial to visualize them together and compare the situation of different countries from different points of view.
The tool is very easy to download, install (there is even the online version) and use.
The offered visualizations are simple and relatively easy to understand even for people without any previous knowledge about the data.

The range of users for this tool is wide.
In general, we can say that the tool is designed for the grand public, for everybody which is curious in the exploration of social/political/historical/economical trends for different countries or continents in the world.
The tool is also perfect for teachers who want to present the world situation (and its evolution over time) using concrete data and effective visualizations 
(Gapminder's website has an entire section dedicated to teachers\footnote{\url{http://www.gapminder.org/for-teachers/}}).

Each indicator in Gapminder has a provides a detailed description, the sources, the confidence about the measures, and describes how data are collected, aggregated and normalized.
However, this information is not explicit in the tool, but is only available from the Gapminder website\footnote{\url{https://www.gapminder.org/data/}}.
Gapminder is not designed for professional economists, analysts and researchers, since they usually need to analyze this kind of information in their work.


\subsection{Purpose}
The purpose of the visualization is to present a picture of the global world situation from many different points of view.
The tool allows to easily compare many different indicators, analyze the situation of different countries in the world, study the historical evolution of the global world situation and predict trends for the next years.

On one side, Gapminder can be used to confirm the known.
The tool can be used to both confirm or discard hypothesis about the world current and past situation.
For example, one could think that wars influence the life expectation of a country.
Gapminder can be used to check the life expectations during the first and second world wars of countries in different continents and compare their situations:
the visualization will confirm that the countries directly involved in the wars had a remarkable drop in life expectance.
The tool allows to answer both generic question (e.g. the distribution of values for a certain indicator over the world or correlations between different indicators) and specific questions (e.g. countries with the highest value for a given indicator).

On the other side, the tool can be used to discover the unknown.
The user can select and combine different indicators and analyze their evolution over time.
During this process, interesting patterns can be visualized.
One can notice, for example, that there is a sudden drop in the population life expectation in both Ukraine and Russia in 1933 and investigate to what it is due.
One can then change the visualized indicators and further explore interesting aspects.
