\section{Description of Findings}
\label{sec:findings}

Gapminder allows to answer many questions about the global world situation and its evolution over time.
We will try to give an answer to some of them:
\begin{itemize}
    \item What is the life expectation of the world population?
    \item What is the global trend of the average income?    
    \item What is the world situation in terms of population, average income, life expectation?
\end{itemize}


\subsection{Life Expectancy}
We can answer this question by using a chart of type ``maps'' and changing both the indicator shown as color and size to life expectancy.
We also reduce the size scale to avoid too many overlapping.
Each bubble represents a country and the position of the bubble its geographical location.
The resulting visualization is shown in \cref{fig:life-expectation}.
We use the most recent available data of $2015$.

\begin{figure}[h]
	\centering
	\includegraphics[width=0.95\columnwidth]{figures/life-expectancy}
	\caption{Life expectancy in $2005$ for different countries of the world.}
	\label{fig:life-expectation}
\end{figure}

We can immediately notice groups of bubbles with different colors.
\begin{itemize}
	\item Bubbles in europe are red or dark orange, which indicates an average life expectation between $75$ (orange) and $85$ (red) years.	Countries in east europe tend to have a slightly lower life expectation that countries in the center and west.
	\item Countries in asia tend to have shorter life expectancy, between $65$ and $75$ years. There are however some exceptions: Japan, South Korea and Singapore have higher life expectations (respectively $83$, $81$ and $83$ years), while Afghanistan has a much shorter one, only $54$ years.
	\item New Zealand and Australia have high life expectations, while the small countries in the Pacific Ocean have quite much lower ones (between $60$ and $65$).
	\item Bubbles in Africa range from green to yellow / light orange. This indicates life expectations between about $50$ and $65$ years. The situation is slightly better in the very north: life expectation here is between $70$ and $75$ years.
	\item Countries in america go from orange to red. Life expectation is quite high in the north (about $80$ years) and in the south (about $75$ years) and lower in the center (from $70$ to $75$ years).
\end{itemize}


\subsection{Income Trend}
We can answer this question by using a chart of type ``lines''.
The default visualization already shows the income per person for countries in different regions,
we only add some countries to the visualized one to have a better overview of the global situation.
The visualization is shown in \cref{fig:income-trend}.

In general, income tends to grow in all regions in the world.
Countries in europe, america and rich countries in asia had an almost constant growth from $1850$ to $2015$, with only some small drops around $1920$ and $1945$ (probably due to the World Wars).
Countries in asia such as China and India did not have a significant growth until the second half of the $20^{th}$ century:
after that moment, the growth is very fast, much faster than europe or america.
Countries in africa show different situations:
some of them are growing (eg. Egypt or Nigeria), others had a growth in the first half of the $20^{th}$ century, followed by a decline.
The situation is quite complex and would require some more complex analysis which is out of the scope of this document.

\begin{figure}[h]
	\centering
	\includegraphics[width=0.95\columnwidth]{figures/income-trend}
	\caption{Trend of the average income per person for different countries of the world.}
	\label{fig:income-trend}
\end{figure}


\subsection{World Situation}
We can answer this question by using a chart of type ``bubbles''.

\begin{figure}[h]
	\centering
	\includegraphics[width=0.95\columnwidth]{figures/world-situation}
	\caption{World situation from the point of view of income, life expectation and population size.}
	\label{fig:world-situation}
\end{figure}
