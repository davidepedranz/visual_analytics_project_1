\section{Data Description}
\label{sec:data}

The data used by Gapminder are public and available for free online \footnote{\url{http://www.gapminder.org/data/}}.

\subsection{Data Attributes}

Each dataset represents an indicator measured over the entire world and different time periods and has thus $3$ dimensions: time, geographical location and value of the indicator. Since the $3$ indicators we consider for this assignment share the first $2$ dimensions (time and country), we have a total of $5$ dimensions visualized by the tool.

\paragraph{Time}
Time is measured in years and is clearly a discrete attribute.
It is possible to state if $2$ given years are the same of not by simply comparing their number.
The ``less than'' $<$ relationship of the set of natural numbers $\mathbb{N}$ defines a total order between years.
It is also possible to sum or subtract a natural number for a given year and the result is still a year.
The datasets contain points for the time range $1800$ to $2015$, for a total of $215$ years.

\paragraph{Geographical Location}
The geographical location is expressed in term of countries, grouped into world regions.
Since it does not exists a natural ordering of countries, the attribute type is categorical.
Gapminder has data about $244$ different countries in the world, grouped in $5$ regions (which approximate the world's division into continents).

\paragraph{Average Income}
The average income is expressed as the gross domestic product per person, adjusted for differences in purchasing power (measured in international dollars).
The attribute is clearly quantitative.
Values range from $300$ to about $150.000$.

\paragraph{Life Expectation}
The life expectation is expressed in years.
The attribute is an average and is thus a continuous values. 
Values range from $1$ to about $85$.

\paragraph{Population Size}
The population size is expressed in number of inhabitants.
This is clearly a natural number, so the population size attribute type is discrete:
The normal equality, ordering and sum defined in $\mathbb{N}$ apply also for the natural numbers.
Although it is possible to compare ratios between population sizes, we can not say that the attribute type is continuous (quantitative). 
The population size ranges from $100$ to about $1.400.000.000$ inhabitants.


\subsection{Data Model}


Each dataset in Gapminder can be represented as a table:
the x-axis represent the time in years, the y-axis represent the country and the data in each cell represent the indicator.

An alternative representation of the data is a set of tuples
each tuple has two dimensions that represent time and country respectively, and an extra dimension for each indicator.

Not all datasets in Gapminder are complete:
most indicators are defined for a small time period (few years) and have missing data for some country and / or some particular year.
