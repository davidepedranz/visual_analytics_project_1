\section{Data Description}
\label{sec:data}

\subsection{Data Attributes}
Each observation in Gapminder is the value of an indicator measured for a specific country in the world in a given year.
The country can be seen as an independent variable, since it is not influences by the indicators (e.g. the average income does not determine what it the country).
All measured indicators are dependent variables, since they refer to a specific country.
In addition, indicators change over time.
The dataset can thus be seen as a 4-dimensional dataset, where the first dimension (country) does not change over time, while the other dimensions (income, life expectation and population size) are time dependent.
The geographical location of countries (i.e. the region of the world to which they belong) should not change over time, so this does not represent an extra dimension (there is a functional dependence between the country and the geographical location as a region of the world).

\paragraph{Geographical Location}
The geographical location is expressed in term of countries, grouped into world regions.
Since it does not exists a natural ordering of countries, the attribute type is categorical.
Gapminder has data about $244$ different countries in the world, grouped in $4$ regions (which approximate the world's division into continents).

\paragraph{Average Income}
The average income is expressed as the gross domestic product per person, adjusted for differences in purchasing power (measured in international dollars).
The attribute is clearly quantitative.
Values range from $300$ to about $150.000$.

\paragraph{Life Expectation}
The life expectation is expressed in years.
The attribute is an average and is thus a continuous values. 
Values range from $1$ to about $85$.

\paragraph{Population Size}
The population size is expressed in number of inhabitants.
This is clearly a natural number, so the population size attribute type is discrete:
the normal equality, ordering and sum defined in $\mathbb{N}$ apply also for the natural numbers.
Although it is possible to compare ratios between population sizes, we can not say that the attribute type is continuous (quantitative). 
The population size ranges from $100$ to about $1.400.000.000$ inhabitants.

\paragraph{Time}
Time is measured in years and is clearly a discrete attribute.
It is possible to state if $2$ given years are the same of not by simply comparing their number.
The ``less than'' $<$ relationship of the set of natural numbers $\mathbb{N}$ defines a total order between years.
It is also possible to sum or subtract a natural number for a given year and the result is still a year.
The dataset contains points for the time range $1800$ to $2015$, for a total of $215$ years.


\subsection{Data Model}

\subsubsection{Internal Representation}
\label{subsubsec:internal_representation}
Each indicator in Gapminder is internally represented as a table:
the x-axis contains the time in years, the y-axis the country, and the data in each cell the value measured for the indicator in the given year and country.

Not all indicators in Gapminder have the same size.
Most of them are defined for a small time period (few years) and have missing data for some country and / or some particular year.
Some indicator are not measured for some countries.
Missing data corresponds to empty cells in some of these tables.

\subsubsection{Alternative Representation}
An alternative representation of the data is a set of tables (one for each year) \texttt{dataset\_<year>}, where \texttt{<year>} is replaced by the year to which the table refers.
Each table has $4$ columns: \texttt{country}, \texttt{income}, \texttt{life\_expectation} and \texttt{population\_size}.
The column \texttt{country} is the primary key.
Each row in the dataset represents the situation of a country, measured in terms for the $3$ indicators described above for the year specified in the table's name.
In other words, each table represents the projection of the entire dataset for a single year.
The same representation would also apply to a bigger subset of indicators: we would simply add an extra column for each considered indicator.

Since all tables have the same structure, they can be combined in a cube:
the first dimensions represents the countries, the second dimension contains all indicators measured for the countries and the last dimension present time.
The projection of the cube along the third dimension (i.e. filtering the data for a specific year) results in the tables described in the previous paragraph.
The projection along the second dimension (i.e. selecting a single indicator) gives the representation described in \cref{subsubsec:internal_representation}. 
The projection along the first dimension (i.e. fixing a country) represents the evolution over time of all indicators for a country and can be seen as a table with the indicators on the x-axis and time in the y-axis.

Additionally, countries are grouped into regions.
This can be represented with an extra table \\ \texttt{countries} with $2$ columns: \texttt{country} (primary key) and \texttt{region}.
The column \texttt{country} of each table \texttt{dataset\_<year>} is a foreign key for \texttt{countries.country}.
