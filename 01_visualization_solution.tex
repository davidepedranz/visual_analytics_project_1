\section{Application Domain and Visualization Solution}
\label{sec:visualization_solution}

The visualization solution analyzed in this assignment is Gapminder tool\footnote{\url{https://www.gapminder.org}}.
Gapminder is a tool that allows to explore a dataset with about $500$ indicators measured around the entire world over different time periods.
Each observation in the dataset represents the value measured for one of the indicators for a specific country in the world in a given year.
Gapminder provides data for all (or almost all countries) in the world.
Countries are organized in groups by geographical region.

Indicators are organized in hierarchical categories (economy, education, environment, health, population, etc.) and each category contains one or more dimensions.
Example of indicators available in Gapminder are:
\begin{itemize}
    \item child mortality rate;
    \item income per person;
    \item CO2 per capita.
\end{itemize}
It is also possible to load and visualize additional indicators provided by the user.

The tool allows to visualize and compare the dimensions in the dataset using the following type of charts:
\begin{itemize}
    \item ``bubbles'' (bubble charts);
    \item ``maps'';
    \item ``mountains'' (area charts);
    \item ``rankings'' (bar charts);
    \item ``lines'' (multivariable line charts).
\end{itemize}

\vspace{3mm}

From the large number of available indicators, I have selected the followings for this assignment:
\begin{itemize}
    \item time (expressed in year);
    \item geographical location (country, continent);
    \item average income per person (GDP /capita, PPP\$ inflation-adjusted);
    \item average life expectancy (years);
    \item total population size.
\end{itemize}

Gapminder is available in $2$ versions: an online\footnote{\url{https://www.gapminder.org/tools/}} and a desktop\footnote{\url{https://www.gapminder.org/tools-offline/}} one.
Both versions seems to be similar to each other, but in this assignment I focus on the latter.
