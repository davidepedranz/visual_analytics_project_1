\section{Application Domain and Visualization Solution}
\label{sec:visualization_solution}

The visualization solution analyzed in this assignment is Gapminder tool \footnote{\url{https://www.gapminder.org}}.
Gapminder is a tool that allows to explore about $500$ different datasets related to various indicators measured over the entire world and over different time periods.
The datasets are organized in hierarchical categories (economy, education, environment, health, population, etc.) and each category contain one or more dataset.
Example of datasets available in Gapminder are:
\begin{itemize}
    \item child mortality rate;
    \item income per person;
    \item CO2 per capita.
\end{itemize}
It is also possible to load and visualize external datasets.
The tool allows to visualize and compare these indicators using the following type of charts:
bubbles, maps, mountains, rankings and lines.

From the large number of available datasets, I have selected the followings for this assignment:
\begin{itemize}
    \item time (expressed in year);
    \item geographical location (country, continent);
    \item average income per person (GDP /capita, PPP\$ inflation-adjusted);
    \item average life expectancy (years);
    \item total population size.
\end{itemize}

Gapminder is available in $2$ versions: an online \footnote{\url{https://www.gapminder.org/tools/}} and a desktop \footnote{\url{https://www.gapminder.org/tools-offline/}} one.
Both versions seems to be similar to each other, but in this assignment I focus on the latter.
