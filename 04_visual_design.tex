\section{Visual Design Study}
\label{sec:visual_design}

\subsection{Type of visualization}
Gapminder is an InfoVis for the following reasons:
\begin{itemize}
    \item Data is time-dependent, multidimensional (we consider only $5$ dimensions for this assignment, but the complete dataset of Gapminder has over $500$ dimensions) and non-spatial.
    \item Data attributes have different types: some are categorical (eg. countries), some are discrete (eg. time in years), some are quantitative (eg. average income). 
    \item Not every point is associated to a measured: different datasets are defined over different time periods and some data is missing; it is not possible to find a point between any pair of points since the some attributes are categorical or discrete.
    \item The visualization is interactive and has the scope to look for correlations between different indicators about the world situation.
\end{itemize}


\subsection{Visual encoding}
Gapminder offers $5$ different types of chart, called respectively: bubbles, maps, mountains, rankings and lines.
Since each chart uses visual variables in different ways, we analyze each of them separately.


\subsubsection{Bubbles}
Gapminder's ``bubbles'' is a bubble chart with animations.
We consider the particular instance of bubble chart with the income on the x-axis, the life expectation on the y-axis and the population size encoded as the bubbles size; this is the default visualization offered by Gapminder when opening the software.

The visual variables being used are: position, size, color, shapes (text labels) and animations.

\vspace{0.5cm}
TODO: shall I put a screenshot here?


\paragraph{Position}
The position is used to encode the value of $2$ indicators: the position along the x-axis encodes the first indicator (in our case the income), the position along the y-axis the second one (in our case the life expectation).
Since both indicators are quantitative, each point in the chart is in a bijective relation with a pair of possible values for the indicators.

Each point in the bubble chart represents a country.
The encoding is very good, since it uses a single visual variable to represent $2$ dimensions of the dataset.
Position is a very powerful visual variable: it is associative, selective, ordinal and quantitative.
The user will thus:
\begin{itemize}
    \item Perceive each point independently of the other visual variables used. This allows to use other variables (eg. size or color) to encode additional dimensions of the data.
    \item Perceive each point as different from each other. This allows to easily distinguish countries from each other on the chart.
    \item Be able to easily order the points. This allows to easily tell which countries have lowest and highest values for one indicator and order the countries based on some indicator.
    \item Be able to easily compute ratios between the values of the indicators encoded by each point. This allows to compare the situation of pairs of different countries from the point of view of the $2$ indicator.
\end{itemize}

Additionally, the position visual variable is used in the colormap for the world's regions, as explained in colors paragraph.
% TODO: we can force the tool to use categorical data for x-axis / y-axis !!! BAD
% Nevertheless, Gapminder allows the user to select other kinds of attributes as well, which might lead to sub-optimal representations.


\paragraph{Size}
The size is used to encode the third indicator, in our case the population size.
In particular, the value is encoded as the area of the bubble.

Size is selective, ordinal and quantitative.
The user can thus:
\begin{itemize}
    \item Easily perceive countries with different populations as different from each other.
    \item Easily order countries by population size.
    \item Easily compute ratios between the size of the populations of different countries.
\end{itemize}

However, the tool does not visualize a legend for the sizes of the bubble.
This makes difficult to perform the inverse mapping from area of the bubble to size of the population by just looking at the bubble chart.
The value is visualized on the right side of the screen only if the user moves the mouse over a bubble.

The encoding of the population sizes is overall clear, but it can be improved.
On the one hand, the user is able to distinguish and compare the population of different countries.
On the other hand, it is quite complex to visualize the real values of the populations.


\paragraph{Color}
\label{paragraph:bubbles-color}
The color can be used to encode any of the available dimensions of the dataset.
In our case, the tool uses the color to encode the world region to which each country belongs.
Gapminder shows a colormap in the top right corner of the windows:
the colormap is visualized a small stylized planisphere where each continent if filled with the color that corresponds to its region.
The tool differentiate $4$ world regions: americas, africa, europe and russia, asia and oceania.

The colormap is categorical and uses different color hues for each different value.
The visualization uses the following $4$ colors: green, blue, yellow, red.
The choice of the color is good, since they are all distinct.
There is no way to change the colormap or force a different mapping.
However it does not take into account colorblind people that might have difficulties in distinguish green and red.

Each bubble has a black border.
This makes it easier to identify a bubble and distinguish it from the other ones.

The color of a bubble together with the colormap allows to match each country with a specific location on the planisphere.
In other words, this allows the user to approximately locate the geographical position of each county.

The encoding if clear: the $4$ choses colors are very different from each other, so it is very easy to distinguish point that corresponds to countries in different regions.
The inverse mapping from color to region of the world is also pretty straightforward, since there are only $4$ possible regions.

The visualization uses a white background, with a big gray text label in the center (to visualize the year, see the animation paragraph).
This is a good design choice, since it is a neutral color that does not interfere with the visualization and creates a high contrast with the colors used for the bubbles.

%, but it is possible to select any other attribute.
% The tool automatically recognize the type of the attribute and uses an appropriate encoding.
% In case of categorical data, the tool generates a legend for the colors: each color is associated with a different categorical values.
% In case of qualitative data, the tool generates a continuos rainbow colormap, ranging from violet to represent the minimum value to red that to represent the maximum.
% It is possible to choose between the linear or a logarithmic scale using the dedicated menu;
% by default the tool uses a linear scale.


\subsubsection{Maps}

\subsubsection{Mountains}

\subsubsection{Rankings}

\subsubsection{Lines}


\subsection{Improvements}
Bullet list of points to discuss:
\begin{itemize}
    \item add a legend for the size of the bubbles
    \item add information about the 3 indicators in the tooltip of the bubble (proximity gestalt law)

    \item colormap... not easy to read for colorblind people... possibility to choose a different colormap for colorblind people, since there are only 4 colors
\end{itemize}
